% ------------------------------------------------------------------------
% config/cmds/cmds_meta.tex
%
% Set meta-commands (mostly for maths mode).
% ------------------------------------------------------------------------

% Add maths environments.

%% Environment 'multline' with 'quad' space as padding.  See the content
%% of this template for an example of its usage.
\makeatletter
    %
    \newcounter{quadcnt}
    %
    \newenvironment{mlines}[1][2]{%
        \setcounter{quadcnt}{#1}%
        \start@multline\st@rredfalse%
        \begingroup%
            \foreach \n in {1,...,\value{quadcnt}}{\quad}%
            \ignorespaces%
        \endgroup
    }{%
        \begingroup%
            \foreach \n in {1,...,\value{quadcnt}}{\quad}%
            \ignorespaces%
        \endgroup
        \iftagsleft@
            \@xp\lendmultline@
        \else
            \@xp\rendmultline@
        \fi
        \ignorespacesafterend
    }
\makeatother


% Define bracketing operator/function maker.  Code recycled from
% package 'physics'.  Used to define some commands in 'cmd_maths.tex'.

\makeatletter
    \let\originalleft\left
    \renewcommand{\left}{\mathopen{}\mathclose\bgroup\originalleft}
    %
    \let\originalright\right
    \renewcommand{\right}{\aftergroup\egroup\originalright}
    %
    \DeclareDocumentCommand\oper{s m}{%
        \IfBooleanTF#1{%
            \operatorname{#2}\!%
        }{%
            \operatorname{#2}\!{\ifnum\z@=`}\fi\@arguments%
        }%
    }
    \DeclareDocumentCommand\func{s m}{%
        \IfBooleanTF#1{%
            #2%
        }{%
            \mathop{}\!#2{\ifnum\z@=`}\fi\@arguments%
        }%
    }
    % Flexibly automatically bracket with (), [], {} or ||.
    \DeclareDocumentCommand\@arguments{
        t\big t\Big t\bigg t\Bigg g o d() d||
    }{
        % Handle manual override of sizing.
        \IfBooleanTF{#1}{%
            \let\ltag\bigl \let\rtag\bigr
        }{%
            \IfBooleanTF{#2}{%
                \let\ltag\Bigl \let\rtag\Bigr
            }{%
                \IfBooleanTF{#3}{%
                    \let\ltag\biggl \let\rtag\biggr
                }{%
                    \IfBooleanTF{#4}{%
                        \let\ltag\Biggl \let\rtag\Biggr
                    }{%
                        \let\ltag\left \let\rtag\right
                    }%
                }%
            }%
        }%
        % Handle actual bracketing.
        \IfNoValueTF{#5}{%
            \IfNoValueTF{#6}{%
                \IfNoValueTF{#7}{%
                    \IfNoValueTF{#8}{%
                        ()
                    }{%
                        \ltag\lvert{#8}\rtag\rvert
                    }%
                }{%
                    \ltag(#7\rtag) \IfNoValueTF{#8}{}{|#8|}
                }%
            }{%
                \ltag[#6\rtag]
                \IfNoValueTF{#7}{}{(#7)}
                \IfNoValueTF{#8}{}{|#8|}
            }%
        }{%
            \ltag\lbrace#5\rtag\rbrace
            \IfNoValueTF{#6}{}{[#6]}
            \IfNoValueTF{#7}{}{(#7)}
            \IfNoValueTF{#8}{}{|#8|}
        }%
        \ifnum\z@=`{\fi}
    }
\makeatother


% Add myriad digit grouping.  This is only relevant to Chinese speakers
% (and probably Korean and Japanese) who group digits every 4 digits, e.g.
% an/a American/modern British billion is '10,0000,0000'.  Code recycled
% from my StackExchange question: https://tex.stackexchange.com/questions/414783/is-it-possible-to-re-group-every-4-digits-in-a-number-with-siunitx-package.

\ExplSyntaxOn
    \NewDocumentCommand{\groupfour}{O{\mkern1mu} m}{
        \digit_groupfour:nn { #1 } { #2 }
    }

    \tl_new:N \l_digit_groupfour_separator_tl
    \tl_new:N \l_digit_groupfour_number_tl

    \cs_new_protected:Nn \digit_groupfour:nn {
        \tl_set:Nn \l_digit_groupfour_separator_tl { {#1} }
        \tl_set:Nx \l_digit_groupfour_number_tl { #2 }
        \tl_replace_all:Nnn \l_digit_groupfour_number_tl { ~ } { }
        \tl_reverse:N \l_digit_groupfour_number_tl
        \regex_replace_all:nnN
            { [0-9]{4} }  % search
            { \0 \c{l_digit_groupfour_separator_tl} }  % replace
            \l_digit_groupfour_number_tl  % token list
        \tl_reverse:N \l_digit_groupfour_number_tl
        \regex_replace_once:nnN
            { \A \c{l_digit_groupfour_separator_tl} }
            { }
            \l_digit_groupfour_number_tl
        \tl_use:N \l_digit_groupfour_number_tl
    }
\ExplSyntaxOff


% Add linebreak-at-comma compatible maths mode command.  This helps with
% line breaking for a long inline maths formula. Code recycled from
% https://tex.stackexchange.com/questions/1959/allowing-line-break-at-in-inline-math-mode

\newcommand{\splitatcommas}[1]{%
    \begingroup
    \ifnum\mathcode`,="8000
    \else
        \begingroup\lccode`~=`, \lowercase{\endgroup
            \edef~{%
                \mathchar\the\mathcode`, \penalty0 \noexpand%
                \hspace{0pt plus 1ex}%
            }%
        }\mathcode`,="8000
    \fi
    #1%
    \endgroup
}
