% ------------------------------------------------------------------------
% config/profiles/core.tex
%
% Main stylesheet for laying out thesis.  See class 'memoir' and LaTeX
% documentations for details.  Modify as appropriate.
% ------------------------------------------------------------------------

% Styles and Layouts
% ----------------------

% Set document-wide tolerance and penalties.  Tweak as appropriate.

% \brokenpenalty=50
% \clubpenalty=9996
% \displaywidowpenalty=1602
% \predisplaypenalty=100
% \postdisplaypenalty=1549
% \widowpenalty=9999
\hfuzz=1pt


% Set the layout of contents tables/lists.

\setpnumwidth{2.0em}
\setrmarg{2.0em plus 1fil}

\setlength{\cftbeforesectionskip}{0.1\baselineskip}
\setlength{\cftbeforetableskip}{0.5\baselineskip plus 0.5\baselineskip}
\setlength{\cftbeforefigureskip}{0.5\baselineskip plus 0.5\baselineskip}

\setlength{\cfttablenumwidth}{3.2em}
\setlength{\cftfigurenumwidth}{3.2em}

\renewcommand{\cftdotsep}{2.0}


% Set the styles of the document hierarchy (chapters, sections etc.).

\setsecnumdepth{subsection}  % numbering depth

%% Set the layout of chapter title pages.

\copypagestyle{chapter}{plain}
\makeoddfoot{chapter}{}{}{}
\makeevenfoot{chapter}{}{}{}

%% Set chapter heading style (named 'providence' in this template).

\renewcommand*{\chaptitlefont}{\hdfont\bfseries\HUGE}

\makeatletter
    \newsavebox{\chapicon@chapter}
    %
    \newcommand\chapicon@chapter@marker{%
        \savebox\chapicon@chapter{%
            \resizebox{!}{1cm}{%
                \fboxsep=2.5pt%
                \colorbox{argent!200}{\vbox to 0.8em %
                    { \vfil\hbox to 0.8em {%
                        \hfil\color{white}\bfseries\thechapter\hfil
                    } \vfil}%
                }%
            }%
        }%
        \usebox{\chapicon@chapter}%
    }
    %
    \newcommand\chapicon@chaptermarker{%
        \savebox\chapicon@chapter{\chapicon@chapter@marker}%
        \makebox[1cm][\align]{\usebox\chapicon@chapter}%
    }
    %
    \makechapterstyle{providence}{%
        \renewcommand\printchaptername{}%
        \renewcommand\printchapternum{%
            \if@twoside%
                \checkoddpage%
                \ifoddpage%
                    \raggedleft%
                    \def\align{r}%
                \else%
                    \raggedright%
                    \def\align{l}%
                \fi%
            \else%
                \hfill%
                \def\align{r}%
            \fi%
            \null\chapicon@chaptermarker%
        }%
        \renewcommand\printchapternonum{%
            \if@twoside%
                \checkoddpage%
                \ifoddpage%
                    \raggedleft%
                \else%
                    \raggedright%
                \fi%
            \else%
                \hfill%
            \fi%
        }%
        \renewcommand\printchaptertitle[1]{%
            \if@twoside%
                \checkoddpage%
                \ifoddpage%
                    \flushright%
                \else%
                    \flushleft%
                \fi%
            \else%
                \flushright%
            \fi%
            \chaptitlefont##1\par%
        }%
        \renewcommand\afterchaptertitle{\vspace{25mm}}%
    }
\makeatother

\chapterstyle{providence}

%% Set section heading style.

\setsecheadstyle{\hdfont\bfseries\Large}
\setsubsecheadstyle{\hdfont\bfseries\large}


% Set main matter style.

%% Set document-wide spacings.

\setSingleSpace{1.1}
\SingleSpacing  % also apply to package 'natbib' command ``bibpreamble``

\setlength{\parindent}{4ex}
\setlength{\textfloatsep}{15pt plus 2pt}

%% Set typical page layouts.

\clearmark{section}

\makepagestyle{formal}
\makeatletter
    \makeoddhead{formal}{%
    }{%
        \sffamily\normalsize%
        \addfontfeature{Scale=0.9,LetterSpace=13}%
        \spaceskip1.5ex%
        \rightmark%
    }{%
        \sffamily\normalsize%
        \thepage%
    }
    \makeevenhead{formal}{%
        \sffamily\normalsize%
        \thepage%
    }{%
        \sffamily\normalsize%
        \addfontfeature{Scale=0.9,LetterSpace=13}%
        \spaceskip1.5ex%
        \leftmark%
    }{}
    \makeoddfoot{formal}{}{}{}
    \makeevenfoot{formal}{}{}{}
\makeatother

\makeatletter
    \makepsmarks{formal}{
        \createmark{chapter}{both}{nonumber}{\@chapapp}{}
    }
\makeatother

%%% Merge 'formal' style and 'empty' style (for pages with only floats)
%%% into 'publisher' style.
\mergepagefloatstyle{publisher}{formal}{empty}

%% Set footnote style.

\feetatbottom
\feetbelowfloat

\newlength{\footmarkspan}
\setlength{\footmarkspan}{\footparindent}
\setlength{\footmarkwidth}{0pt}
\setlength{\footmarksep}{0pt}
\setlength{\footparindent}{\parindent}
\footmarkstyle{%
    \makebox[\footmarkspan][l]{\textsuperscript{#1}}%
}


% Miscellaneous Elements
% --------------------------

% Define environment 'inset' for standalone inserts.

\newenvironment{inset}{%
    \cleardoublepage
    \thispagestyle{empty}
    \vspace*{0.25\textheight}\par
    \begin{center}
    \hdfont\bfseries\color{\inking}  % use theme colour
}{
    \par
    \vfill
    \pgfornament[color=\inking,width=7mm]{24}\par  % use theme colour
    \vspace{0.15\textheight}\par
    \end{center}
}
\newcommand{\insetpage}[1]{\begin{inset}#1\end{inset}}  % shortcut command


% Define special command to enlarge or shrink the usable area on certain
% pages to suppress artefacts and warnings related to 'overfull' and
% 'underfull' '\vbox'.

\newcommand{\footsaviour}[1][]{%
    \enlargethispage{#1\baselineskip}%
}


% Define wrapper command 'lsmode' for landscape full-page figures.

\newlength\normalwd
\newcommand{\lsmode}[1]{%
    \setlength{\normalwd}{\textwidth}%
    \afterpage{\begin{landscape}#1\end{landscape}}%
}


% Add URL formatting command, which deals with line breaking issues
% typically found for URLs.  This removes prefixes such as 'http://',
% 'www' etc.  A 'fontawesome' icon glyph is used to indicate a web link
% (unless the starred vertsion is used). Code recycled from
% https://tex.stackexchange.com/questions/3033/forcing-linebreaks-in-url
% and https://tex.stackexchange.com/questions/285823/multi-substring-replace.
% E.g. ``\weblink{www.google.co.uk}`` is shown as an weblink icon and
% the hyperlinked text 'google.co.uk'.

\expandafter\def\expandafter\UrlBreaks\expandafter{%
    \UrlBreaks%  % save the current one
    \do\a\do\b\do\c\do\d\do\e\do\f\do\g%
    \do\h\do\i\do\j\do\k\do\l\do\m\do\n%
    \do\o\do\p\do\q\do\r\do\s\do\t%
    \do\u\do\v\do\w\do\x\do\y\do\z%
    \do\A\do\B\do\C\do\D\do\E\do\F\do\G%
    \do\H\do\I\do\J\do\K\do\L\do\M\do\N%
    \do\O\do\P\do\Q\do\R\do\S\do\T%
    \do\U\do\V\do\W\do\X\do\Y\do\Z%
    \do0\do1\do2\do3\do4\do5\do6\do7\do8\do9%
    \do=\do/\do.\do:\do\*\do\-\do\~\do\'\do\"\do\\%
}  % none other technique works if ``\href`` is used instead of ``\url``

%TC:macro \weblink [ignore]
\NewDocumentCommand\weblink{s m}{%
    \IfBooleanTF#1{%
        \def\prefix{}%
    }{%
        \def\prefix{%
            \raisebox{-0.5pt}{\scriptsize\,\faExternalLink}%
            \nobreak\hspace{1mm}\nobreak%
        }%
    }%
    \saveexpandmode\noexpandarg%
    \StrSubstitute{#2}{https://}{}[\wlink]%
    \expandafter\StrSubstitute\expandafter{\wlink}{http://}{}[\wlink]%
    \expandafter\StrSubstitute\expandafter{\wlink}{www.}{}[\wlink]%
    \restoreexpandmode%
    \href{#2}{\prefix\nolinkurl{\wlink}}%
}


% Add a visual break between paragraphs to indicate a change of topic.
% Usually fleuron symbols are used, but note the Unicode code for fleuron
% symbols differ between different typefaces.  For package 'libertine',
% the fleuron glyph used is 'U+E001' here.  The starred variant reflects
% the glyph.

\NewDocumentCommand\topicbreak{s}{%
    \IfBooleanTF#1{%
        \fancybreak{%
            \vspace*{1em}%
            \raisebox{0.0\baselineskip}{\reflectbox{\symbol{"E001}}}%
            \vspace{1em}%
        }
    }{%
        \fancybreak{%
            \vspace*{1em}%
            \raisebox{0.0\baselineskip}{\symbol{"E001}}%
            \vspace{1em}%
        }
    }%
}


% Redefine equation references in bracketed form.  An equation is referred
% to inside round brackets with optional pre-text, e.g. ``\beqref[see ]
% {eq:equation}`` may show '(see eq.~5)' partly hyperlinked.

%TC:macro \beqref [option:word,ignore]
\makeatletter
    \let\originalref\ref
    \newcommand{\beqref}[2][]{%
        \begingroup%
            \let\ref\@refstar%
            (#1\hyperref[#2]{eq.~\originalref{#2}})%
        \endgroup%
    }
\makeatother


% Define 'oneside'/'twoside'-dependent command.

%TC:macro \oneortwosidecmd [ignore,ignore]
\makeatletter
\newcommand{\oneortwosidecmd}[2]{
    \if@twoside%
        #2%
    \else%
        #1%
    \fi%
}
\makeatother
