% ------------------------------------------------------------------------
% config/pkgs/pkg_rc.tex
%
% Package configurations.  Tweak various settings to achieve desired
% typesetting outcomes.  See individual package documentation for
% further instructions.
% ------------------------------------------------------------------------

% Package: amsmath
% --------------------

\allowdisplaybreaks


% Package: bibentry
% ---------------------

\nobibliography*


% Package: caption
% --------------------

\DeclareCaptionLabelSeparator{quadsep}{.\quad}

\captionsetup[table]{position=above}
\captionsetup{
    skip=1.5em,
    labelsep={quadsep},
    labelfont={sf,bf,small},
    textfont={small}
}

% Set numbering format for consecutively numbered figures in a series.
\renewcommand{\theContinuedFloat}{(\alph{ContinuedFloat})}


% Package: cleveref
% ---------------------

% Capitalise only tables and figures. Abbreviate only bracketed figure
% and equation references.

\crefname{figure}{Figure}{Figures}
\crefname{table}{Table}{Tables}

\creflabelformat{equation}{#2(#1)#3}
\crefmultiformat{equation}{%
    equations~#2(#1)#3%
}{ and~#2(#1)#3}{, #2(#1)#3}{ and~#2(#1)#3}
\crefrangeformat{equation}{equations~#3(#1)#4--#5(#2)#6}

\newcommand{\crefrangeconjunction}{--}


% Add local, temporary switches between bracketed, abbreviated references
% (``\abbrvcrefnames``) and full references (``\fullcrefnames``) by
% issuing ``\bref`` combining the two.

\DeclareRobustCommand{\fullcrefnames}{%
    \crefname{chapter}{chapter}{chapters}%
    \crefname{section}{section}{sections}%
    \crefname{subsection}{subsection}{subsections}%
    \crefname{figure}{Figure}{Figures}%
    \crefname{equation}{equation}{equations}%
    \crefname{subequation}{equation}{equations}%
    \crefformat{equation}{##2equation~(##1)##3}%
    \crefformat{subequation}{##2equation~(##1)##3}%
    \crefmultiformat{equation}{%
        equations~##2(##1)##3%
    }{ and~##2(##1)##3}{, ##2(##1)##3}{ and~##2(##1)##3}%
    \crefmultiformat{subequation}{%
        equations~##2(##1)##3%
    }{ and~##2(##1)##3}{, ##2(##1)##3}{ and~##2(##1)##3}%
    \crefrangeformat{equation}{equations~##3(##1)##4--##5(##2)##6}%
    \crefrangeformat{subequation}{equations~##3(##1)##4--##5(##2)##6}%
}
\DeclareRobustCommand{\abbrvcrefnames}{%
    \crefname{chapter}{\S\!}{\S\S\!}%
    \crefname{section}{\S\!}{\S\S\!}%
    \crefname{subsection}{\S\!}{\S\S\!}%
    \crefname{figure}{Fig.}{Figs.}%
    \crefname{equation}{eq.}{eqs.}%
    \crefname{subequation}{eq.}{eqs.}%
    \crefformat{equation}{##2eq.~##1##3}%
    \crefformat{subequation}{##2eq.~##1##3}%
    \crefmultiformat{equation}{%
        eqs.~##2##1##3%
    }{ and~##2##1##3}{, ##2##1##3}{ and~##2##1##3}%
    \crefmultiformat{subequation}{%
        eqs.~##2##1##3%
    }{ and~##2##1##3}{, ##2##1##3}{ and~##2##1##3}%
    \crefrangeformat{equation}{eqs.~##3##1##4--##5##2##6}%
    \crefrangeformat{subequation}{eqs.~##3##1##4--##5##2##6}%
}

%% Format references as '[pre-text][post-text]{cite_tag}', e.g.
%% ``\bref[see ][ for details]{eq:field equation,eq:probability}`` may
%% render as '(see eq.~5 and~6 for details)' hyperlinked.
%TC:macro \bref [option:word,option:word,1]
\NewDocumentCommand{\bref}{O{} O{} m}{%
    {({#1}\abbrvcrefnames\cref{#3}{#2})}%
    \fullcrefnames%
}


% Package: datetime
% ---------------------

\setdefaultdate{\yyyymmdddate}

\renewcommand{\dateseparator}{.}


% Package: enumitem
% ---------------------

\setlist[enumerate]{
    label={\arabic*)},
    topsep=1pt,
    itemsep=1pt,
    parsep=\parskip,
    labelsep=1ex,
    labelindent=\parindent,
    leftmargin=*
}
\setlist[itemize]{
    label={\raisebox{1pt}{\fontsize{7}{7}\selectfont\(\blacksquare\)}},
    topsep=1pt,
    itemsep=1pt,
    parsep=\parskip,
    labelsep=1ex
}


% Package: floatrow
% ---------------------

\floatsetup[table]{capposition=top}


% Package: fnpct
% ------------------

\makepagenote

\setfnpct{punct-after}


% Package: geometry
% ---------------------

\geometry{inner=40mm,outer=23.5mm,top=28mm,bottom=20mm}  % default margins


% Package: glossaries-extra
% -----------------------------

\makeglossaries

% Determine whether to disable hyperlinks for certain glossary categories.
\glssetcategoryattribute{topic}{nohyper}{true}
\glssetcategoryattribute{notation}{nohyper}{true}

% Use specific formats for certain glossary categories.
\setabbreviationstyle[acronym]{short}
\setabbreviationstyle[code]{short-sc}

% Adjust glossary table spacing for abbreviations and notations.
\newglossarystyle{abbrstyle}{%
    \setglossarystyle{long}%
    % Set width of glossary entry name columns.
    \let\namecolumnwidth\relax%
    \newlength\namecolumnwidth%
    \setlength\namecolumnwidth{4.5em}%
    % Set column separation.
    \let\columnsep\relax%
    \newlength\columnsep%
    \setlength\columnsep{1.8em}%
    % Declare column types named '\Nm' and '\Ds'.
    \newcolumntype{\Nm}{>{\raggedright\arraybackslash}p{\namecolumnwidth}}%
    \newcolumntype{\Ds}{%
        >{\raggedright\arraybackslash}%
        p{\linewidth-\namecolumnwidth-\columnsep-2.4em}%
    }%
    % Declare glossaties as 'longtable' with customised columns.
    \renewenvironment{theglossary}{%
        \begin{longtable}[l]{\Nm @{\hspace{\columnsep}} \Ds}%
    }{%
        \end{longtable}%
    }%
}
\newglossarystyle{symbstyle}{%
    \setglossarystyle{long}%
    % Set width of glossary entry name columns.
    \let\namecolumnwidth\relax%
    \newlength\namecolumnwidth%
    \setlength\namecolumnwidth{4.5em}%
    % Set column separation.
    \let\columnsep\relax%
    \newlength\columnsep%
    \setlength\columnsep{1.8em}%
    % Declare column types named '\Nm' and '\Ds'.
    \newcolumntype{\Nm}{>{\raggedleft\arraybackslash}p{\namecolumnwidth}}%
    \newcolumntype{\Ds}{%
        >{\raggedright\arraybackslash}%
        p{\linewidth-\namecolumnwidth-\columnsep-2.4em}%
    }%
    % Declare glossaties as 'longtable' with customised columns.
    \renewenvironment{theglossary}{%
        \begin{longtable}[l]{\Nm @{\hspace{\columnsep}} \Ds}%
    }{%
        \end{longtable}%
    }%
}

% Allow only the first instance used in text to be hyperlinked
% to glossary.
\renewcommand*{\glslinkcheckfirsthyperhook}{%
    \ifglsused{\glslabel}{%
        \setkeys{glslink}{hyper=false}%
    }{}%
}

% Change typesetting for individual glossary elements.
\renewcommand{\glsnamefont}[1]{\textbf{#1}}
% \renewcommand{\glsxtrfullsep}[1]{~}  % disabled due to 'overfull \hbox'
                                       % difficulty


% Package: graphics
% ---------------------

\graphicspath{{attachments/graphics/}{attachments/figures/}}


% Package: hyperref
% ---------------------

\hypersetup{
    hidelinks,
    breaklinks=true,
    colorlinks=false,
    linktoc=all,
    bookmarksdepth=2,
    draft=false
}

\hypersetup{
    pdftitle={Grand Thesis Title},
	pdfauthor={Author McAuthor},
	pdfsubject={Unified Theory of Everything},
    pdfkeywords={%
        add thesis keyword; add additional thesis keyword%
    },
    pdfcreator={XeLaTeX with hyperref in the memoir class},
	pdfinfo={PubType={Doctoral Thesis}},
    pdfstartview={XYZ null null 1.00},
    pdfstartpage=1
}


% Package: letterine
% ----------------------

\renewcommand{\LettrineTextFont}{}  % annul default font settings
                                    % for initial letters


% Package: microtype
% ----------------------

\microtypesetup{draft=false}


% Package: natbib
% -------------------

\setcitestyle{numbers,square,nonamebreak,citesep={,},notesep={}}

\setlength{\bibsep}{0.5\baselineskip plus 0.5\baselineskip}

\renewcommand{\bibpreamble}{\SingleSpacing\small}
\renewcommand{\bibnumfmt}[1]{{%
    \addfontfeature{Numbers={Proportional,Lining}}%
    \hphantom{[999]}%  % reserve sufficient horizontal spacing up to the
                       % width of '[999]'
    \llap{[{\footnotesize#1}]}%  % collapse onto reserved
                                 % horizontal spacing
}}


% Reset footnote-specific citation number font.

%% Declare Boolean switch for whether inside/outside footnotes.
\newif\ifinfootnote
\infootnotefalse

%% Switch citation number font depending on whether
%% inside/outside footnotes.
\let\oldfootnote\footnote
\renewcommand{\footnote}[1]{\infootnotetrue\oldfootnote{#1}\infootnotefalse}
\renewcommand{\citenumfont}[1]{{%
    \ifinfootnote%
        \tiny%
    \else%
        \footnotesize%
    \fi%
    \addfontfeature{Numbers={Proportional,Lining}}%
    #1%
}}


% Add 'texcount' directives for some package 'natbib' commands.

%TC:macro \cite [option:word,option:word,ignore]
%TC:macro \citetalias 1
%TC:macro \citepalias [ignore]


% Package: silence
% --------------------

% Try to suppress innocuous warnings.
\WarningFilter{babel}{The following font is not a babel standard family}
\WarningFilter{babel}{The following fonts are not babel standard families}
\WarningFilter{caption}{Unused \captionsetup}
\WarningFilter{hyperref}{Token}
\WarningFilter{latexfont}{Font shape}
\WarningFilter{latexfont}{Size substitution}
\WarningFilter{latexfont}{Some font shapes were not available, defaults substituted}


% Package: siunitx
% --------------------

% Tweak and add as appropriate.

\sisetup{separate-uncertainty=true}

\DeclareSIUnit\deg{deg}
\DeclareSIUnit\parsec{pc}
\DeclareSIUnit\solarmass{\ensuremath{M_\odot}}


% Package: tikz
% -------------------

% Load 'tikz' libraries as appropriate.
\usetikzlibrary{arrows.meta,calc,svg.path}


% Package: xcolor
% -------------------

% Add brand-specific colours.  Add more as appropriate.
\definecolor{orcid}{HTML}{A6CE39}  % ORCID icon

% Base this template's colour scheme on shades of grey (designed to
% be almost monochrome).
\definecolor{slategrey}{HTML}{708090}  % theme colour used in the
                                       % command ``\inking``
\definecolor{argent}{HTML}{C0C0C0}  % shading colour for document elements
\definecolor{platinum}{HTML}{E5E4E2}  % shading colour for 'tikz' diagrams

% Offer additional colours for the user to consider.  None below
% is used in this template.
\definecolor{tyrianpurple}{HTML}{66023C}
\definecolor{imperialblue}{HTML}{005A92}
\definecolor{royalblue}{HTML}{002366}

% Define the theme colour used in e.g. the title page, epigraph etc.
\newcommand{\inking}{slategrey!175}
