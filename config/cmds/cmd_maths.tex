% ------------------------------------------------------------------------
% config/cmds/cmds_maths.tex
%
% Maths-mode commands. The TeX primitive ``$`` mostly appears here.
% ------------------------------------------------------------------------

% -------------------
%     MATHEMATICS
% -------------------

% Diacritics-Related
% ----------------------

\let\primeT\prime  % old thick prime
\renewcommand{\prime}{\primeS}  % thin prime

\let\narrowbar\bar
\renewcommand{\bar}[1]{\widebar{#1}}

\let\narrowtilde\tilde
\renewcommand{\tilde}[1]{\widetilde{#1}}

\let\narrowcheck\check
\renewcommand{\check}[1]{\widecheck{#1}}

% Adjust height of letter under diacritic.
% E.g. ``\lowerh{-0.2ex}{\widehat{P}}``.
\newcommand{\lowerh}[2]{\raisebox{#1}{$#2$}}

% Add shortcut/alias commands.
\newcommand{\placeholder}{{%
    \vcenter{\hbox{%
        \protect\scalebox{0.5}{$\mkern2mu\bullet\mkern2mu$}%
    }}%
}}

\newcommand{\est}[1]{\widehat{#1}}
\newcommand{\fid}[1]{\breve{#1}}
\newcommand{\filter}[1]{\tilde{#1}}


% Relations and Delimiters
% ----------------------------

\DeclarePairedDelimiter\floor{\lfloor}{\rfloor}

\DeclareDocumentCommand\given{o m m}{%  % conditional probability, e.g.
                                        % ``\given[\Big]{A}{B}``
	\IfNoValueTF{#1}{{%
		\argopen.\mkern-2mu{#2}\mkern2mu%
        \middle|%
        \mkern2mu{#3}\mkern-2mu\argclose.%
	}}{%
	    #2\,#1|\,#3%
	}%
}

\DeclareDocumentCommand\divergence{o m m}{%  % (KL) divergence, e.g.
                                             % ``\divergence[\Big]{f}{g}``
	\IfNoValueTF{#1}{{%
		\argopen.\mkern-2mu{#2}\mkern2mu%
        \middle\|%
        \mkern2mu{#3}\mkern-2mu\argclose.%
	}}{%
	    #2\,#1\|\,#3%
	}%
}

% Add/modify shortcut/alias commands.
\let\geqflat\geq
\renewcommand{\geq}{\geqslant}

\let\leqflat\leq
\renewcommand{\leq}{\leqslant}

\newcommand{\deq}{=}
\newcommand{\below}{\mathord{<}\,}


% Differentials and Operations
% --------------------------------

% Define field measure (upright 'D').
\newcommand{\dD}{\mathop{}\!\mathrm{D}}
\DeclareDocumentCommand\Dm{o g d()}{%
	\IfNoValueTF{#2}{%
		\IfNoValueTF{#3}{%
		    \dD\IfNoValueTF{#1}{}{^{#1}}%
		}{%
		    \mathinner{\dD\IfNoValueTF{#1}{}{^{#1}}\argopen[#3\argclose]}%
            \mkern1mu%
		}%
	}{%
	    \mathinner{\dD\IfNoValueTF{#1}{}{^{#1}}#2} \IfNoValueTF{#3}{}{[#3]}%
	    \mkern1mu%
	}%
}

% Add/modify shortcut/alias commands.
\newcommand{\dda}[1]{\dd[2]{#1}}
\newcommand{\ddd}[1]{\dd[3]{#1}}

% Fine tune the transposition symbol.
\makeatletter
\newcommand*{\transT}{{\mathpalette\@transT{}}}
\newcommand*{\@transT}[2]{%
    \raisebox{\depth}{$\m@th#1\intercal$}%
}
\makeatother

% Add/modify shortcut/alias commands.
\newcommand{\conj}[1]{#1^\ast}
\newcommand{\trans}[1]{#1^\transT}
\newcommand{\herm}[1]{#1^\dagger}

\newcommand{\conv}{\mathbin{*}}  % convolution operation


% Function(al)s and Operators
% -------------------------------

% Note the starred variants (e.g. ``\Prob*``) of the following commands
% are for function(al)s shown without argument, e.g. 'f' rather than
% 'f(x)', or for occasions when automatic resizing of brackets is not
% desired.  See the content of this template for examples of the
% following commands.

\NewDocumentCommand\Prob{s}{%  % probability (mass), e.g. ``\Prob(X)``
    \IfBooleanTF#1{%
        \func*{\mathbb{P}}
    }{%
        \func{\mathbb{P}}%
    }%
}
\NewDocumentCommand\PDF{s o}{%  % probability density, e.g. ``\PDF(x)``
    \IfBooleanTF#1{%
        \IfNoValueTF{#2}{%
            \func*{\raisebox{0.1ex}{$\mathbb{p}$}}%
        }{%
            \func*{\raisebox{0.1ex}{$\mathbb{p}$}_{#2}}%
        }%
    }{%
        \IfNoValueTF{#2}{%
            \func{\raisebox{0.1ex}{$\mathbb{p}$}}%
        }{%
            \func{\raisebox{0.1ex}{$\mathbb{p}$}_{#2}}%
        }%
    }%
}

\NewDocumentCommand\Exp{s}{%  % expectation, e.g. ``\Exp[X]``
    \IfBooleanTF#1{%
        \oper*{\mathbb{E}}%
    }{%
        \oper{\mathbb{E}}%
    }%
}
\NewDocumentCommand\Var{s}{%  % variance, e.g. ``\Var[X]``
    \IfBooleanTF#1{\oper*{var}}{\oper{var}}%
}
\NewDocumentCommand\Cov{s}{%  % covariance, e.g. ``\Cov[X]``
    \IfBooleanTF#1{\oper*{cov}}{\oper{cov}}%
}
\NewDocumentCommand\Corr{s}{%  % correlation, e.g. ``\Corr[X]``
    \IfBooleanTF#1{\oper*{corr}}{\oper{corr}}%
}

\NewDocumentCommand\Like{s o}{%  % likelihood, e.g. ``\Like[subscript](x)``
    \IfBooleanTF#1{
        \IfNoValueTF{#2}{%
            \func*{\mathscr{L}}%
        }{%
            \func*{\mathscr{L}_{#2}}%
        }%
    }{%
        \IfNoValueTF{#2}{%
            \func{\mathscr{L}}%
        }{%
            \func{\mathscr{L}_{#2}}%
        }%
    }%
}
\NewDocumentCommand\Post{s}{%  % posterior, e.g. ``\Post(x)``
    \IfBooleanTF#1{%
        \func*{\mathscr{P}}%
    }{%
        \func{\mathscr{P}}%
    }%
}
\NewDocumentCommand\Prior{s}{%  % prior, e.g. ``\Prior(x)``
    \IfBooleanTF#1{%
        \func*{\itpi}%
    }{%
        \func{\itpi}%
    }%
}

\NewDocumentCommand{\KL}{m m}{%  % KL divergence, e.g. ``\KL{f}{g}``
    \func{D_\mathrm{KL}}(\divergence{#1}{#2})%
}

\NewDocumentCommand\legendre{s m}{%  % Legendre polynomials, e.g.
                                     % ``\legendre{\ell}(\mu)``
    \IfBooleanTF#1{%
        \func*{\mathcal{L}_{#2}}%
    }{%
        \func{\mathcal{L}_{#2}}%
    }%
}
\NewDocumentCommand\sphY{s m}{%  % spherical harmonic, e.g.
                                 % ``\sphY{\ell m}(x)``
    \IfBooleanTF#1{%
        \func*{\tensor{Y}{_{#2}}}%
    }{%
        \func{\tensor{Y}{_{#2}}}%
    }%
}
\NewDocumentCommand\sphYc{s m}{%  % spherical harmonic complex conjugate
    \IfBooleanTF#1{%
        \func*{\tensor*{Y}{^\ast_{#2}}}%
    }{%
        \func{\tensor*{Y}{^\ast_{#2}}}%
    }%
}

\NewDocumentCommand\dirac{s}{%  % Dirac delta, e.g. ``\dirac(x)``
    \IfBooleanTF#1{%
        \func*{\delta^\mathrm{(D)}}%
    }{%
        \func{\delta^\mathrm{(D)}}%
    }%
}
\newcommand{\kron}{\delta^\mathrm{(K)}}  % Kronecker delta


% Miscellaneous
% -----------------

% Typeset matrices in bold italic (relying on package 'isomath').

\newcommand{\mat}[1]{\mathsfbfit{#1}}


% (Re-)Define symbols.

\let\oldforall\forall
\let\forall\undefined
\DeclareMathOperator{\forall}{\oldforall}

\renewcommand{\emptyset}{\varnothing}
\renewcommand{\C}{\mathbb{C}}  % complex numbers
\newcommand{\R}{\mathbb{R}}  % real numbers
\newcommand{\set}{\qty}  % a set, e.g. ``\set{a, b, c}``

\renewcommand{\Re}{\operatorname{Re}}
\renewcommand{\Im}{\operatorname{Im}}

\renewcommand{\pi}{\mkern1mu\uppi\mkern1mu}
\newcommand{\e}{\mkern1mu\mathrm{e}\mkern1mu}
\newcommand{\im}{\mkern1mu\mathrm{i}\mkern1mu}

\newcommand{\sparallel}{\varparallel}  % slanted parallel

\newcommand{\diff}[1]{{\upDelta #1}}


% ---------------------
%     MISCELLANEOUS
% ---------------------

% Add custom maths commands.
\newcommand{\measurement}[3]{{#1}^{#2}_{#3}}  % for demo purpose
